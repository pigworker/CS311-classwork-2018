\documentclass{article}
\usepackage{a4wide}
\usepackage{alltt}
\usepackage{amsmath}
\usepackage{stmaryrd}
\usepackage{latexsym}
\usepackage{pig}

\newcommand{\fbx}[1]{\framebox{\ensuremath{#1}}}
\newcommand{\la}[2]{\backslash\;#1\;\texttt{->}\;#2}
\newcommand{\pl}[2]{\texttt{(}#1\;\texttt{+}\;#2\texttt{)}}

\begin{document}
\title{LAM --- Semantic Variations}
\author{Neil Ghani and Conor McBride}
\maketitle

Our grammar of $\lambda$-terms is given by
\begin{verbatim}
<term>
  ::= <var>
    | <term> <term>
    | \ <var> -> <term>
    | <number>
    | (<term> + <term>)
\end{verbatim}
If we were real purists, we would omit numbers and addition:
we include them to represent `ordinary' computation, where
the job of the $\lambda$-terms is to route data to whichever
computation is appropriate.


\section{Small Step Reduction}

We write $f$, $s$ and $t$ for terms, and $m$ and $n$ for numerical constants.
Our judgement form is
\[
  \fbx{s\leadsto t}
\]
with $s$ as an input and $t$ as an output.

The key axioms which do the work are $\beta$-reduction and numerical addition.
\[
  \Axiom{(\la x{t[x]})\;s \leadsto t[s]}
  \qquad
  \Axiom{\pl mn \leadsto m+n}
\]
They are not, however, enough: before you make a reduction step, you have to find a redex! We add \emph{contextual closure} rules:
\[\begin{array}{c}
  \Rule{f\leadsto f'}
  {f\;s \leadsto f'\:s}
  \qquad
  \Rule{s\leadsto s'}
  {f\;s \leadsto f\:s'}
  \\
  \Rule{t[x]\leadsto t'[x]}
  {\la x{t[x]} \leadsto \la x{t'[x]}}
  \\
  \Rule{s \leadsto s'}
  {\pl st \leadsto \pl{s'}t}
  \qquad
  \Rule{t \leadsto t'}
  {\pl st \leadsto \pl s{t'}}
\end{array}\]
The contextual closure rules are very far from arbitrary: they are generated from the grammar, with one rule for every way to put a subterm inside a term. Their impact is to allow $\beta$-reduction and numerical addition \emph{anywhere} inside a term. These rules are highly nondeterministic: when there are multiple redexes, they do not tell you which one to choose.


\section{Big Step Reduction}

Here is one way to give a big step semantics for LAM. Before we can write the rules, we need to say `what success looks like': to what \emph{values} are we
trying to compute the terms?

In our language of functions and numbers, let us say that a function is a value if it is ready to be applied, and a number is a value if it is ready to be added.
\begin{verbatim}
<value>
  ::= \ <var> -> <term>
    | <number>
\end{verbatim}
Note that for functions we make no demands on the body of the abstraction.
Let us write $u$ and $v$ for values. Our judgement form is
\newcommand{\lbs}[2]{#1\,\Downarrow\,#2}
\[\fbx{\lbs tv}
\]
with $t$ as input and $v$ as output.
We have only four rules. Two tell us how to stop
\[
  \Axiom{\lbs{\la x{t[x]}}{\la x{t[x]}}}
  \qquad
  \Axiom{\lbs nn}
\]
and two tell us how to go.
\[
  \Rule{\lbs{f}{\la x{t[x]}}\quad \lbs{t[s]}v}
  {\lbs{f\:s}v}
  \qquad
  \Rule{\lbs sm\quad \lbs tn}
  {\lbs{\pl s t}{m+n}}
\]
There are several things to notice about this semantics:
\begin{itemize}
\item It does not give \emph{every} term a value. (Think about which terms do not get a value, and why. There are several kinds of `bad'.)
\item It has a deterministic evaluation strategy.
\item It never computes inside the body of an abstraction: rather, it leaves the body untouched until the function's argument turns up.
\item It is still reliant on substitution.
\item A function's arguments are not evaluated until after substitution: this is a win if the bound variable is unused, but unfortunate if it is used a lot.
\end{itemize}


\section{Closures and Environments}

There is no particularly good reason why values should be a subset of terms.
They could be something apart. Here is a way to represent function values more efficiently, getting rid of substitution.
\begin{verbatim}
<val>
  ::= [<env>] <var> -> <term>
    | <number>

<env>
  ::=
    | <env>, <var> = <val>
\end{verbatim}
Our function values are now given as structures called \emph{closures} which pack up an \emph{environment} along with an abstraction. An environment is a mapping of variables to values. A closure captures the situation where we are not quite ready to evaluate the body of an abstraction: we know the values of all the \emph{free} variables, but not the value of the bound variable.

We write $u$ and $v$ for `vals' and $\gamma$ for environments. Our judgement form now takes an environment, to explain what the free variables mean.
\[
  \fbx{\gamma | \lbs tv}
\]
We expect every free variable to have an entry in the environment. We make the further assumption that all the free variables have different names: this is achievable by $\alpha$-conversion (but there is a better way).

\newcommand{\tto}{\,\texttt{->}\:}
We have one rule per construct.
\[\begin{array}{c}
  \Axiom{\gamma | \lbs{\la xt}{[\gamma]x \tto t}}
  \qquad
  \Axiom{\gamma | \lbs nn}
  \qquad
  \Axiom{\gamma,x=v,\gamma' | \lbs xv}
    \\
  \Rule{\gamma | \lbs f{[\gamma']x \tto t} \quad
    \gamma | \lbs su \quad
    \gamma',x=u | \lbs tv}
    {\gamma | \lbs{f\:s}v}
    \quad
    \Rule{\gamma | \lbs sm \quad \gamma | \lbs tn}
      {\gamma | \lbs{\pl st}{m+n}}
\end{array}\]

In a real implementation, we get rid of variable \emph{names} entirely.
We replace names variable usage sites by numbers called \emph{de Bruijn indices},
locally distinguished from numerical constants by a \# marker,
which tell us how many $\lambda$s to jump over before we find the one which bound the variable. That is
\[
  \la f{\la x{f\:(f\:x)}}
\]
becomes
\[
  \la{}{\la{}{\#1\:(\#1\:\#0)}}
\]
A de Bruijn index is just the thing to tell us \emph{where} in $\gamma$, now an
\emph{array} of values (with 0 the rightmost index), is the value we want.


\section{Thunks}

\end{document}