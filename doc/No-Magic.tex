\documentclass{article}
\usepackage{a4wide}
\usepackage{amsfonts}
\usepackage{stmaryrd}
\usepackage{alltt}
\usepackage{pig}

\begin{document}
\title{There Is No Magic \\
  Type Inference for LAM}
\author{Neil Ghani and Conor McBride}
\maketitle

\newcommand{\Nat}{\mathbb{N}}
\newcommand{\fbx}[1]{\framebox{\ensuremath{#1}}}
\newcommand{\NT}[1]{\langle #1 \rangle}
  
\section{Introduction}

Our typing rules for LAM have been given as follows:
\[\begin{array}{c}
    \Axiom{\Gamma_0,x:\sigma,\Gamma_1 \vdash x:\sigma}\qquad
    \Rule{\Gamma,x:\sigma\vdash t:\tau}
    {\Gamma\vdash \lambda x \to t : \sigma \to \tau}\qquad
    \Rule{\Gamma\vdash f:\sigma\to\tau\quad \Gamma\vdash s:\sigma}
    {\Gamma\vdash f\:s : \tau}
    \\
    \Axiom{\Gamma\vdash n:\Nat} \qquad
    \Rule{\Gamma\vdash s:\Nat\quad\Gamma\vdash t:\Nat}
      {\Gamma\vdash (s+t) : \Nat}
\end{array}\]

One should rightly be suspicious of these rules. On the one hand,
if types are thought of as an \emph{input} to be \emph{checked},
where on earth does $\sigma$ come from in the rule for application?
On the other hand, if types are thought of as an \emph{output} to be
\emph{synthesized}, where on earth does the $\sigma$ come from in
the rule for $\lambda$? Either way, the `right' type is seemingly conjured
from thin air!

What really happens when computers give types to LAM terms?


\section{Normal Forms}

If we write down the grammar of $\beta$-normal terms, something
interesting happens
\[\begin{array}{lrl}
    \NT{normal} & ::= & \lambda \NT{var} \to \NT{normal} \\
    & | & \NT{number} \\
    & | & \NT{neutral} \medskip \\
    \NT{neutral} & ::= & \NT{var} \\
   & | &  \NT{neutral} \;  \NT{normal} \\
   & | &  (\NT{neutral} +  \NT{normal}) \\
   & | &  (\NT{number} +  \NT{neutral}) \\
\end{array}  \]
We have carefully ensured that `neutral' terms are \emph{stuck}
for some reason --- always, ultimately, that there is a free
variable getting in the way of computation. The `normal' forms
are things that can be value-like --- numbers or $\lambda$-abstractions ---
but they are prevented from standing in any place that could trigger
a reduction.

We \emph{can} give a sensible algorithmic system for typing this syntax.
The trick is to treat types as an input for normals but an output for neutrals.
Let us write $s,t,n$ for normals and $e,f$ for neutrals. Our judgement forms are
\[
  \Gamma \vdash \tau \ni t \qquad \Gamma \vdash e\in\sigma
\]
We have stolen the set theoretic $\in$ symbol and exploited its lack of
symmetry! When we \emph{check} a type, we write the type \emph{before} the
term, because it is given in advance. Here are the checking rules.
\[
  \Rule{\Gamma,x:\sigma \vdash \tau \ni t}
  {\Gamma\vdash \sigma\to\tau \ni \lambda x\to t}
  \qquad
  \Axiom{\Nat\ni n}
  \qquad
  \Rule{\Gamma\vdash e\in\sigma\quad \sigma = \tau}
       {\Gamma\vdash \tau \ni e}
  \]
The third rule says that if we are checking a type $\tau$ for a neutral,
first synthesize the type $\sigma$ of the neutral, then test if the type you
get is the type you want. But note also the first rule, in which the type
at which we check a $\lambda$ tells us what type to give for its bound
variable in the context.

Ultimately, the synthesis rules work on the basis that variables types are indeed
known by the context.
\[\begin{array}{c}
  \Axiom{\Gamma_0,x:\sigma,\Gamma_1\vdash x\in\sigma} \qquad
  \Rule{\Gamma\vdash f\in \sigma\to\tau \quad \Gamma\vdash \sigma\ni s}
    {\Gamma\vdash f\:s\in \tau}
    \\
  \Rule{\Gamma\vdash e\in\Nat\quad \Gamma\vdash \Nat\ni t}
  {\Gamma\vdash (e + t) \in \Nat} \qquad
  \Rule{\Gamma\vdash e\in\Nat}
    {\Gamma\vdash (n+e)\in\Nat}
\end{array}  \]

See that in the application rule, $\sigma$ comes from the type of the
function and is then used to check the argument.

So, we have no need of any magic when our terms are in normal form. If we
restrict ourselves to programs that we do not want to run, we have no
problem. So we still have a problem.

What we need type information for is to \emph{justify} doing computation!


\section{Type Annotations}

One way to fix the problem by brute force and allow programs which compute
is to extend the grammar with explicit type annotations, exactly where the
information is missing.
\[
  \NT{neutral} ::= \ldots \;|\; \NT{normal}:\NT{type}
\]
We get the rule
\[
  \Rule{\Gamma\vdash \tau\ni t}
    {\Gamma\vdash t:\tau \in \tau}
\]
but we have to adjust the small-step semantics to account for the fact that
computations now have type annotations. Neutral terms may now compute.
\[
  (\lambda x\to t : \sigma\to\tau)\;s \leadsto
  t[s:\sigma/x]:\tau \qquad
  (n:\Nat + t) \leadsto (n + t:\Nat) \qquad
  (n + n':\Nat) \leadsto (n + n'):\Nat
\]
(In the last reduction rule, the grammar tells you that the $+$ on the right
cannot be a piece of syntax standing between two numerical values, so it
indicates that we actually do the arithmetic at that point!)

The only reduction rule for normal terms is
\[
  t : \tau \leadsto t
\]
which says `a normal form which not being \emph{used} does not need a
type annotation'.

Everywhere computation happens, we can see the type at which it is happening.
In particular, the $\beta$ rule does computation at a function type
$\sigma\to\tau$ which may trigger further computation, but only at
smaller types, $\sigma$ and $\tau$. Fundamentally, that is why all well
typed programs in this language terminate.


\section{Definitions}

In ordinary programming, we don't just write one great big expression: we
make definitions, naming components we intend to use later, and we use them
just by invoking their names.

Another way to avoid the need for magic is to give types when we make
definitions. Instead of putting annotations directly into neutrals, let us
consider
\[
  \NT{neutral} ::= \ldots \;|\; \mathtt{let}\;x=\NT{normal}:\NT{type}\;\mathtt{in}\; \NT{neutral}
\]
with the typing rule
\[\Rule{\Gamma\vdash \sigma\ni s\quad \Gamma,x:\sigma\vdash e\in\tau}
       {\Gamma\vdash \mathtt{let}\;x=s:\sigma\;\mathtt{in}\;e\in\tau}
\]
Again, we supply just the information that is needed. Programs do not
contain $\beta$-redexes, but they do contain definitions and defined
symbols. To \emph{run} a program, turn it into an ordinary LAM
term by substituting out all the definitions. It will still be well
typed and not go wrong.

That is to say, giving types for definitions, which is good documentation
anyway, should always be enough.


\section{Oh, All Right, There Is Magic, But It's Just Algebra}

As it happens, for a language as simple as LAM, you can get away with
writing no types at all. The method we use is straightfoward:
\begin{enumerate}
\item Extend the grammar of types with \emph{variables} which stand
  for types we don't know yet. (Let's write $?\NT{number}$ for such
  a variable.)
\item Ensure that we can always invent fresh variables to stand for
  new things we don't know. (As we're numbering them, we can always
  remember the lowest number we haven't used yet.)
\item When we don't know a type, we invent a variable to stand for it.
\item When we make demands about types, that tells us how to solve
  for those variables.
\end{enumerate}

Specifically,
\begin{itemize}
\item When inferring a type for $\lambda x\to t$, invent a fresh
  variable for $x$'s type, put that in the context, and hope to
  figure it out later.
\item Every time you have something whose type is a variable $?n$,
  but you want a $\Nat$ (i.e., in the addition rule), congratulations! You now know that
  $?n=\Nat$ and can make that substitution.
\item Every time you have something whose type is a variable $?n$,
  but you want a function type (i.e., demanding a function in the application rule), congratulations! You can make up
  two new type variables $?s$ and $?t$ for the unknown source and target
  types, then solve $?n=?s\to?t$.
\item Every time that you want two types to be \emph{equal} (i.e., when demanding
  that a function's argument has the function's source type), solve the
  equation between the types, and report an error if you cannot.
\end{itemize}

To solve
\begin{itemize}
\item $\Nat = \Nat$ \\ You win!
\item $\sigma\to\tau = \sigma'\to\tau'$ \\
  Solve $\sigma = \sigma'$ and $\tau = \tau'$.
\item $\Nat = \sigma'\to\tau'$ or vice versa \\ You lose!
\item $?n = ?n$ \\ You win! But you don't learn what $?n$ is.
\item $?n = \tau$ or vice versa, if variable $?n$ occurs strictly inside type $\tau$ \\
  You lose! (We cannot come up with any type which is a subformula of itself.)
\item $?n = \tau$ or vice versa, if $?n$ does not occur in $\tau$ \\
  Substitute $\tau$ for $?n$
\end{itemize}

The above algorithm is called \emph{first-order unification}. It was published by Alan
Robinson in 1965. The insight to use it to figure out the missing parts of types was
due to Robin Milner, in the mid-1970s. It has the property that it makes only those
solutions that are forced by the constraints we face. As a result, if any type
variables remain unsolved, it does not matter what type they stand for!

Let us have an example. Infer a type for $\lambda f\to\lambda x\to f\;(f\;x)$.
We're trying to derive
\[
  \Rule{\ldots}
    {\vdash \lambda f\to\lambda x\to f\;(f\;x) : \ldots}
\]
Make up a type for $f$,
\[
  \Rule{\Rule{\ldots}{f:\,?0 \vdash \lambda x\to f\;(f\;x) : \ldots}}
    {\vdash \lambda f\to\lambda x\to f\;(f\;x) : \ldots}
\]
Now make up a type for $x$.
\[
  \Rule{\Rule{\Rule{\ldots}{f:\,?0,x:\,?1\vdash f\;(f\;x) : \ldots}}
        {f:\,?0 \vdash \lambda x\to f\;(f\;x) : \ldots}}
    {\vdash \lambda f\to\lambda x\to f\;(f\;x) : \ldots}
\]
Now, the application rule demands a \emph{function} type for the function. The box marks the thing that is not yet what we need.
\[
  \Rule{\Rule{\Rule{\Axiom{f:\,?0,x:\,?1\vdash f : \fbx{?0}}\quad\Rule{\ldots}{f:\,?0,x:\,?1\vdash f\;x : \ldots}
        }{f:\,?0,x:\,?1\vdash f\;(f\;x) : \ldots}}
        {f:\,?0 \vdash \lambda x\to f\;(f\;x) : \ldots}}
    {\vdash \lambda f\to\lambda x\to f\;(f\;x) : \ldots}
\]
So we make up a function type $?2\to ?3$ and use it to solve $?0$.
\[
  \Rule{\Rule{\Rule{\Axiom{f:\,?2\to?3,x:\,?1\vdash f :\, ?2\to?3}\quad\Rule{\ldots}{f:\,?2\to?3,x:\,?1\vdash f\;x : \ldots}
        }{f:\,?2\to?3,x:\,?1\vdash f\;(f\;x) : \ldots}}
        {f:\,?2\to?3 \vdash \lambda x\to f\;(f\;x) : \ldots}}
    {\vdash \lambda f\to\lambda x\to f\;(f\;x) : \ldots}
\]
In fact, we can fill in a lot of missing types, now.
\[
  \Rule{\Rule{\Rule{\Rule{\Axiom{f:\,?2\to?3,x:\,?1\vdash ?2\to?3}}{f:\,?2\to?3,x:\,?1\vdash f :\, ?2\to?3}\quad\Rule{\ldots}{f:\,?2\to?3,x:\,?1\vdash f\;x :\, ?2}
        }{f:\,?2\to?3,x:\,?1\vdash f\;(f\;x) :\, ?3}}
        {f:\,?2\to?3 \vdash \lambda x\to f\;(f\;x) :\, ?1\to?3}}
    {\vdash \lambda f\to\lambda x\to f\;(f\;x) : (?2\to?3)\to?1\to?3}
\]
We know that we want the remaining application to give us a $?2$, but what have we got?
\[
  \Rule{\Rule{\Rule{\Axiom{f:\,?2\to?3,x:\,?1\vdash f :\, ?2\to?3}\quad
        \Rule{\Axiom{f:\,?2\to?3,x:\,?1\vdash f : \, ?2\to\fbx{?3}}\quad \Axiom{f:\,?2\to?3,x:\,?1\vdash x\,:\fbx{?1}}}{f:\,?2\to?3,x:\,?1\vdash f\;x :\, ?2}
        }{f:\,?2\to?3,x:\,?1\vdash f\;(f\;x) :\, ?3}}
        {f:\,?2\to?3 \vdash \lambda x\to f\;(f\;x) :\, ?1\to?3}}
    {\vdash \lambda f\to\lambda x\to f\;(f\;x) : (?2\to?3)\to?1\to?3}
\]
For the $f\;x$ to be well typed, we need $?1=?2$ and for it to fit where we are putting it, we need $?3=?2$, so we can solve $?1$, then $?3$:
\[
  \Rule{\Rule{\Rule{\Axiom{f:\,?2\to?2,x:\,?2\vdash f :\, ?2\to?2}\quad
        \Rule{\Axiom{f:\,?2\to?2,x:\,?2\vdash f:\,?2\to?2}\quad \Axiom{f:\,?2\to?2,x:\,?2\vdash x:\,?2}}{f:\,?2\to?2,x:\,?2\vdash f\;x :\, ?2}
        }{f:\,?2\to?2,x:\,?2\vdash f\;(f\;x) :\, ?2}}
        {f:\,?2\to?2 \vdash \lambda x\to f\;(f\;x) :\, ?2\to?2}}
    {\vdash \lambda f\to\lambda x\to f\;(f\;x) : (?2\to?2)\to?2\to?2}
\]
We have no solution for $?2$, but no constraints on it, either. It is whatever you want it to be. That is, the type of the $x$ does not matter, but the $f$ has to be a function that maps that type to itself, so that you can do it twice.

Let's have another example, this time of an error: $\lambda x\to x\;x$. We're trying
to derive
\[
  \Rule{\ldots}
       {\vdash \lambda x\to x\;x : \ldots}
\]
so we make up a variable for the type of $x$.
\[
  \Rule{\Rule{\ldots}{x:\,?0\vdash x\;x : \ldots}}
       {\vdash \lambda x\to x\;x : \ldots}
\]
When we deploy the application rule and the variable rule, we learn that $?0$ must be
a function type, so we make one up and substitute it. We now know that the application gives back $?2$
\[
  \Rule{\Rule{\Axiom{x:\,?0\vdash \fbx{?0}} \quad \ldots}{x:\,?0\vdash x\;x : \ldots}}
       {\vdash \lambda x\to x\;x : \ldots}
\qquad\qquad
  \Rule{\Rule{\Axiom{x:\,?1\to?2\vdash x:\,?1\to?2} \quad \ldots}{x:\, ?1\to?2\vdash x\;x : \ldots}}
  {\vdash \lambda x\to x\;x : \ldots}
\qquad\qquad
  \Rule{\Rule{\Axiom{x:\,?1\to?2\vdash x:\,?1\to?2} \quad \ldots}{x:\, ?1\to?2\vdash x\;x :\, ?2}}
  {\vdash \lambda x\to x\;x :\,(?1\to ?2)\to ?2}
\]
The trouble arrives when we check the argument. The variable rule gives us its type.
\[
  \Rule{\Rule{\Axiom{x:\,?1\to?2\vdash x:\,?1\to?2} \quad \Axiom{x:\,?1\to?2\vdash x:\,\fbx{?1\to?2}}}{x:\, ?1\to?2\vdash x\;x :\, ?2}}
  {\vdash \lambda x\to x\;x :\,(?1\to ?2)\to ?2}
\]
For a valid use of the application rule, we must solve
\[
  ?1\to ?2 \;=\; ?1
\]
That is, we need a function type equal to its own source type. That cannot happen, so this term is rejected.


\section{What Have We Learned?}

\begin{enumerate}
\item You do not need to do anything clever to figure out types as long as you are willing to annotate redexes or write types for definitions.
\item You can figure out missing types by introducing variables to stand for unknown types, then solving the constraints that the rules impose on those variables.
\item In our LAM language, the constraints can easily be solved by first-order unification.
\item If in some fancy language, the constraints are just \textsc{too hard} to solve by an algorithm, see 1.
\end{enumerate}


\end{document}